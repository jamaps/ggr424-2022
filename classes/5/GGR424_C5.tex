\documentclass[aspectratio=169]{beamer}
%\usetheme{CambridgeUS}
%\usecolortheme{beaver}

%\usefonttheme{serif}
%\usepackage{helvet}

\usefonttheme{serif}     % Font theme: serif
%\usepackage{ccfonts}     % Font family: Concrete Math
\usepackage[T1]{fontenc} % Font encoding: T1

\setbeamersize{text margin left=42pt,text margin right=42pt} 
\setbeamertemplate{navigation symbols}{}
\setbeamertemplate{itemize items}[default]

\beamertemplatenavigationsymbolsempty

\definecolor{fore}{RGB}{51,51,51}
\definecolor{back}{RGB}{255, 254, 250}
\definecolor{title}{RGB}{ 255, 15, 0}
\definecolor{links}{RGB}{18, 168, 255}

\setbeamercolor{titlelike}{fg=title}
\setbeamercolor{normal text}{fg=fore,bg=back}
\setbeamercolor{alerted text}{fg=title}
\setbeamercolor{itemize item}{fg=title}
\setbeamercolor{enumerate item}{fg=title}
\hypersetup{colorlinks,urlcolor=links}

% for code https://kbroman.org/blog/2013/10/07/better-looking-latexbeamer-slides/
\usepackage{listings}
\definecolor{keywords}{RGB}{255,0,90}
\definecolor{comments}{RGB}{60,179,113}
\lstset{language=Python,
keywordstyle=color{keywords},
commentstyle=color{comments}emph}

% fonts
\usepackage[sc]{mathpazo}


% title info
\title{\textbf{Transportation Networks \& Land Use:}}
\subtitle{\textbf{GGR424 - Transportation Geography \& Planning}}
\author{Jeff Allen}
\institute{University of Toronto}
\date{February 7, 2022}


\begin{document}
	
\begin{frame}
	\titlepage	
\end{frame}



\begin{frame}
	
	\textbf{Land Use}
	
	\textbf{Urban Form}
	
	\textbf{Built Environment}
	
	\textbf{Urban Spatial Structure}
	
\end{frame}





\begin{frame}
	
	In transportation geography and planning, we are usually working with \textbf{vector} data (rather than raster data)
	
	\vspace{2mm}
	
	\textbf{Land Use Data}
	
	\begin{itemize}
		\item What is located where
		\item Usually \textbf{Points} or \textbf{Polygons}
	\end{itemize}

	\vspace{2mm}
	
	\textbf{Network Data}
	
	\begin{itemize}
		\item The spatial patterns of transportation networks 
		\item Usually \textbf{Lines} (and nodes/intersections)
	\end{itemize}
		
\end{frame}






% land use data
	
	


\begin{frame}
	
	\begin{columns}
		\begin{column}{0.5\textwidth}
			
			\textbf{Network} - an interconnected group or system \\
			\vspace{3mm}
			Examples
			\begin{itemize}
				\item Computer network
				\item Social network
				\item Transportation network
				\item Biological network
			\end{itemize} \vspace{3mm}
			Often represented using \textbf{graphs}
		\end{column}
		
		\begin{column}{0.5\textwidth}
			\begin{figure}
				\centering
				\includegraphics[width=0.9\linewidth]{images/internet}
				\label{fig:internet}
			\end{figure}
			\tiny Source: \url{https://en.wikipedia.org/wiki/Network_science}
		\end{column}
		
	\end{columns}
	
\end{frame}




\begin{frame}
	
	\begin{columns}
		
		\begin{column}{0.5\textwidth}
			
			\textbf{Graph}
			\begin{itemize}
				\item Set of \textit{nodes} (also called points or vertices) and \textit{edges} (also called lines or arcs)
				\item $G = (V, E)$
				\item If two nodes have a relationship, then there is an edge linking them
				\item Edges can have weights (e.g. travel time or speed, surface quality, elevation, etc.)
				\item Graphs can be directed or un-directed (e.g. can have one-way relationship)
			\end{itemize}
			
		\end{column}
		
		\begin{column}{0.4\textwidth}
			\begin{figure}
				\centering
				\includegraphics[width=0.9\linewidth]{images/simple_graph}
			\end{figure}
			\tiny Source: \url{https://en.wikipedia.org/wiki/Graph_(discrete_mathematics)}
		\end{column}	
		
	\end{columns}
	
\end{frame}





\begin{frame}
	
	\small Can you walk across all of the seven bridges in K\"onigsberg, without ever repeating a single bridge in the course of one's walk? (Leonhard Euler, 1736)
	
	\begin{figure}
		\centering
		\includegraphics[width=0.9\linewidth]{images/konigsberg_1}
	\end{figure}
	\tiny \tiny Source: \url{https://medium.com/basecs/konigsberg-seven-small-bridges-one-giant-graph-problem-2275d1670a12}
	
\end{frame}





\begin{frame}
	
	\small Representing K\"onigsberg as a graph
	
	\begin{columns}
		
		\begin{column}{0.5\textwidth}
			
			\begin{figure}
				\centering
				\includegraphics[width=1.1\linewidth]{images/konigsberg_2}
			\end{figure}
			
		\end{column}
		
		\begin{column}{0.5\textwidth}
			\begin{figure}
				\centering
				\includegraphics[width=1.0\linewidth]{images/konigsberg_3}
			\end{figure}
		\end{column}
		
	\end{columns}
	\vspace{4mm}
	\tiny \tiny Source: \url{https://medium.com/basecs/konigsberg-seven-small-bridges-one-giant-graph-problem-2275d1670a12}
\end{frame}




\begin{frame}
	
	\begin{figure}
		\centering
		\includegraphics[width=1\linewidth]{images/flight_path_panam_1948}
	\end{figure}
	
\end{frame}




\begin{frame}

	
	\begin{columns}
		\begin{column}{0.5\textwidth}
			GIS applications of network analysis usually pertain to transportation networks.\\
			\vspace{4mm}
			
			Analyzing distances and travel times over \textit{network} space. \\
			\vspace{4mm}
			
			\tiny Source: \url{https://www.openstreetmap.org}
		\end{column}
		\begin{column}{0.5\textwidth}
			
			\begin{figure}
				\centering
				\includegraphics[width=1\linewidth]{images/utsc_roads_network}
			\end{figure}
			
		\end{column}
	\end{columns}
	
	
	
	
\end{frame}






\begin{frame}
	
	
	\begin{columns}
		\begin{column}{0.5\textwidth}
			\textbf{Network Distance}
			
			\begin{itemize}
				\item The distance or travel time between two points, based on the \textit{shortest-path} in a network graph.
				\item Included in many mapping applications and software (e.g. Google Maps, Uber, etc.)
				\item Different than straight-line (e.g. Euclidean) distance
			\end{itemize}
			\vspace{4mm}
			\tiny Source: \url{https://www.openstreetmap.org}
		\end{column}
		\begin{column}{0.5\textwidth}
			
			\small \textit{shortest-path} by walking (22 min, 1.8 km)
			\begin{figure}
				\centering
				\includegraphics[width=1\linewidth]{images/route_utsc_walk}
			\end{figure}
		\end{column}
	\end{columns}

\end{frame}





\begin{frame}

	
	\begin{columns}
		\begin{column}{0.5\textwidth}
			\small \textit{shortest-path} by bicycle (12 min, 2.5 km)
			\begin{figure}
				\centering
				\includegraphics[width=1\linewidth]{images/route_utsc_bike}
			\end{figure}
		\end{column}
		\begin{column}{0.5\textwidth}
			
			\small \textit{shortest-path} by car (8 min, 3.0 km)
			\begin{figure}
				\centering
				\includegraphics[width=1\linewidth]{images/route_utsc_car}
			\end{figure}
		\end{column}
	\end{columns}


\end{frame}



\begin{frame}

	\begin{figure}
		\centering
		\includegraphics[width=0.95 \linewidth]{images/orlando}
	\end{figure}
	\tiny Source: Google Maps (2019)
\end{frame}




\begin{frame}

	\textbf{Isochrones} (iso = equal, chrone = time) - A buffer based on \textit{network} distances or travel times
	
	
	\begin{figure}
		\centering
		\includegraphics[width=0.8\linewidth]{images/STC_buffers}
	\end{figure}

	
\end{frame}



\begin{frame}
	
	
	\begin{figure}
		\centering
		\includegraphics[width=0.8\linewidth]{images/Isochronic_Galton_1881}
	\end{figure}
	\tiny	Source:  Galton,   Francis.   1881.   "On   the   Construction   of   Isochronic   Passage-Charts." Proceedings  of  the  Royal  Geographical  Society  and Monthly Record of Geography 3: 657-658
\end{frame}


\begin{frame}
	
	\begin{figure}
		\centering
		\includegraphics[width=1\linewidth]{images/bike_vs_transit}
	\end{figure}
	
	\tiny Source: Allen, J. - \textit{Using network segments in the visualization of urban isochrones} -  Cartographica - \url{http://jamaps.github.io/docs/allen_2018_isochrones.pdf}
	
\end{frame}



\begin{frame}

	\textbf{Closest Facility Analysis} - finding the nearest location(s) from a set of locations distributed over space \\
	\vspace{3mm}
	Often used in medical and emergency services. 
	\begin{itemize}
		\item e.g. which fire station is closest to a fire
		\item e.g. what is the nearest emergency room
	\end{itemize}
\end{frame}

\begin{frame}

	
	\textbf{Location Allocation}
	\begin{itemize}
		\item Procedures for determining the optimal location for one or more facilities that will service demand from a given set of points across space
		\item Often used in planning new locations of retail, public facilities, distribution centres, etc.
		\item Often use network distances + other data (e.g distributions of population)
	\end{itemize}
	
\end{frame}


\begin{frame}
	
	\begin{figure}
		\centering
		\includegraphics[width=0.9\linewidth]{images/tor_libraries}
	\end{figure}
	
	\tiny Source: \url{https://commons.wikimedia.org/wiki/File:Toronto_Public_Libraries_and_Population_Density.png}
\end{frame}	

\begin{frame}

	
	\textbf{Travelling Salesman}
	
	\begin{itemize}
		\item 	Given a list of locations, and the (network) distances between each pair of locations, what is the shortest possible route that visits each location and returns to the origin point?
		\item e.g. what is the optimal route a salesman can take to visit potential clients in a region 
		\item other applications include planning delivery routes or road trips
	\end{itemize}
	
\end{frame}

\begin{frame}
	\small The optimal road trip visiting 50 cities in the USA
	\begin{figure}
		\centering
		\includegraphics[width=0.9\linewidth]{images/travel_usa}
	\end{figure}
	
	\tiny Source: Randy Olson (2015)  \url{http://www.randalolson.com/2015/03/08/computing-the-optimal-road-trip-across-the-u-s/}
\end{frame}	



\end{document}