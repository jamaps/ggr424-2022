% Don't touch this %%%%%%%%%%%%%%%%%%%%%%%%%%%%%%%%%%%%%%%%%%%
\documentclass[11pt]{article}
\usepackage{fullpage}
\usepackage[left=1in,top=1in,right=1in,bottom=1in,headheight=3ex,headsep=3ex]{geometry}
\usepackage{graphicx}
\usepackage{float}
\renewcommand{\theenumi}{\Alph{enumi})}

\newcommand{\blankline}{\quad\pagebreak[2]}
%%%%%%%%%%%%%%%%%%%%%%%%%%%%%%%%%%%%%%%%%%%%%%%%%%%%%%%%%%%%%%

% Modify Course title, instructor name, semester here %%%%%%%%

\title{\textbf{GGR424: Transportation Geography \& Planning}}
\author{Jeff Allen}
\date{Winter, 2022}

%%%%%%%%%%%%%%%%%%%%%%%%%%%%%%%%%%%%%%%%%%%%%%%%%%%%%%%%%%%%%%

% Don't touch this %%%%%%%%%%%%%%%%%%%%%%%%%%%%%%%%%%%%%%%%%%%
\usepackage[sc]{mathpazo}
\linespread{1.05} % Palatino needs more leading (space between lines)
\usepackage[T1]{fontenc}
\usepackage[mmddyyyy]{datetime}% http://ctan.org/pkg/datetime
\usepackage{advdate}% http://ctan.org/pkg/advdate
\newdateformat{syldate}{\twodigit{\THEMONTH}/\twodigit{\THEDAY}}
\newsavebox{\MONDAY}\savebox{\MONDAY}{Mon}% Mon
\newcommand{\week}[1]{%
	%  \cleardate{mydate}% Clear date
	% \newdate{mydate}{\the\day}{\the\month}{\the\year}% Store date
	\paragraph*{\kern-2ex\quad #1, \syldate{\today} - \AdvanceDate[4]\syldate{\today}:}% Set heading  \quad #1
	%  \setbox1=\hbox{\shortdayofweekname{\getdateday{mydate}}{\getdatemonth{mydate}}{\getdateyear{mydate}}}%
	\ifdim\wd1=\wd\MONDAY
	\AdvanceDate[7]
	\else
	\AdvanceDate[7]
	\fi%
}
\usepackage{setspace}
\usepackage{multicol}
%\usepackage{indentfirst}
\usepackage{fancyhdr,lastpage}
\usepackage{url}
\pagestyle{fancy}
\usepackage{hyperref}
\usepackage{lastpage}
\usepackage{amsmath}
\usepackage{layout}

\lhead{}
\chead{}
%%%%%%%%%%%%%%%%%%%%%%%%%%%%%%%%%%%%%%%%%%%%%%%%%%%%%%%%%%%%%%

% Modify header here %%%%%%%%%%%%%%%%%%%%%%%%%%%%%%%%%%%%%%%%%
\rhead{\footnotesize Final Project | GGR424}
\lhead{\footnotesize Jeff Allen}
%%%%%%%%%%%%%%%%%%%%%%%%%%%%%%%%%%%%%%%%%%%%%%%%%%%%%%%%%%%%%%
% Don't touch this %%%%%%%%%%%%%%%%%%%%%%%%%%%%%%%%%%%%%%%%%%%
\lfoot{}
\cfoot{\small \thepage/\pageref*{LastPage}}
\rfoot{}

\usepackage{array, xcolor}
\usepackage{color,hyperref}
\definecolor{clemsonorange}{HTML}{f00000}
\hypersetup{colorlinks,breaklinks,linkcolor=clemsonorange,urlcolor=clemsonorange,anchorcolor=clemsonorange,citecolor=black}

\setlength{\parindent}{0em}
\setlength{\parskip}{0.8em}

\usepackage{colortbl}
\usepackage{tabularx,ragged2e}
\usepackage{sectsty}


\usepackage{helvet}

\begin{document}
	
	\allsectionsfont{\sffamily}
	
	\section*{\\Transportation Improvement Plan}
	
	
	For the final project in this class, you will propose an intervention, solution, or improvement to a transportation problem you have identified in either your previous assignments, something mentioned in class, or another project of your choosing. 
	
	Your project can vary by mode, geography, and objectives. Both big and small ideas are welcomed. You will be expected to include in your final report some discussion about the practicality/barriers (e.g. politically, economically, etc.) in implementing your proposal in the real world.
	
	Here are a few project ideas:
	\begin{itemize}
		\item Improvements to infrastructure that currently exist in the world, e.g.
		\begin{itemize}
			\item How to improve a specific street (e.g. re-designing College St.)
			\item How to improve a specific transit route (e.g. Line 2 Subway)
			\item How to improve a specific transit station (e.g. St. George Stationo)
		\end{itemize}
		\item Proposing new infrastructure, e.g.
		\begin{itemize}
			\item Plan and design a new transit route (e.g. a BRT on Steeles)
			\item This can be in contrast to a currently proposed route (e.g. how would you design the Ontario Line?)
			\item Plans and designs for how to improve cycling or pedestrian safety and connectivity in a neighbourhood or city
			\item Develop a "transportation master plan" for a neighbourhood or city
		\end{itemize}
		\item Outlining policy recommendations to tackle broader issues
		\begin{itemize}
			\item e.g. policy options to improve transportation equity, sustainability, deliveries, accessibility, or safety, etc. in a neighbourhood, city, or region
		\end{itemize}
	\end{itemize}
	
	A project proposal will be due March 10 \textbf{(5\%)}
	
	The final two classes (March 28 and April 4) will be devoted to short (5 min) presentations \textbf{(5\%)}
	
	The final report will be due April 8 (no extensions) \textbf{(25\%)}
	
	
	\newpage
	
	
	
	
	
	
	\section*{\\Project Proposal}
	
	\textit{Due March 10 at 11:59pm}
	
	Provide a short \textbf{(maximum 500 words, not including any maps, figures, or references)} proposal which, 1) briefly identifies and describes the transportation problem that you will be assessing, and 2) outlines what you plan to do for your project (e.g. creating policy recommendations, maps and GIS analyses, designs like schematic route maps or street cross-sections, etc.)
	
	Please submit your proposal on Quercus as a PDF or Word doc prior to the due date

	\subsubsection*{Marking Guideline}
	
	The proposal will be marked out of \textbf{(5)}:
	
	\textbf{(2)} Problem description
	
	\textbf{(2)} Project plan
	
	\textbf{(1)} Writing quality



	\newpage
	
	
	
	\section*{\\Presentation}
	
	\textit{March 28 or April 4}
	
	\textbf{Please submit your slides as a PDF on Quercus before noon on the day of your presentation}.
	
	You will give a 5 minute presentation to the class outlining your final project.
	
	You do not have to present a "complete" project (especially for those presenting on March 28). It will be more than okay if you provide a background for your project and describe what you're planning on doing going forward.
	
	Keep the text on your slides to a minimum. Since the presentation is rather short, focus on the "big picture" and any key findings/solutions to your transportation problem thus far, rather than delving into small details.
	
	If you have a preference for presenting on either March 28 or April 4, please email me. Preference for presentation dates will be considered on a first-come, first-serve basis. If you don't have a preference, you will be assigned randomly to either date.
	
	
	
	
	\subsubsection*{Marking Guideline}
	
	The presentation will be marked out of \textbf{(5)}:
	
	Your grade will be based on your communication (design of slides and quality of speaking) and storytelling (do you clearly and succinctly describe your project; including background information and your work thus far)
	
	
	
	
	
	\newpage
	
	
	\section*{\\Report}
	
	\textit{Due April 8 at 11:59pm}
	
	The final report should consist of two parts: 1) A shorter section outlining and providing background the transportation problem that you are analyzing. And 2) a longer section detailing your proposed plan, designs, and/or policy recommendations. Include as well some discussion about the practicality/barriers (e.g. politically, economically, etc.) in implementing your proposal in the real world
	
	Feel free to consider using a variety of communication methods in your report; including design sketches, maps, street plans, renderings, and/or photographs.
	
	\textbf{Maximum 10 pages of content including all images and text (single-spaced font allowed)}. Include a title page with your name and student number (not counted towards page count) and a bibliography of all works cited at the end (also not included in your page count). There is no preferred citation style (e.g. APA, MLA, Chicago), just be consistent with whichever you pick throughout your report. 
	
	Please submit your proposal on Quercus as a PDF or Word doc prior to the due date. \textbf{There will be no extensions given for the final report since it is due on the last day allowed by the department for term work}
	
	
	
	
	
	

	\subsubsection*{Marking Guideline}
	
	\textbf{(8)} Communication 
	
	Quality of writing, design of any graphics like maps or schematics if applicable.
	
	\textbf{(5)} Background 
	
	Outlining and providing background to the transportation problem that you are analyzing. This can be an expansion to what you wrote in your proposal. Content will vary depending on your project, but it can include describing its geography, its history, what mode(s) its focused on, specific infrastructure, who does it impact, why is it a problem worth analyzing, etc. Include any maps and figures as well as references and citations where suitable.
	
	\textbf{(12)} Solution 
	
	Describing your proposed plan, designs, and/or policy recommendations. Overall you will be graded on how evidence-based, well-reasoned, and creative your proposed plans/designs/policy recommendations are. This will partly be graded in relation to the problems you outlined in the previous section. 
	
	\vspace{5mm}
	
	
	
	
\end{document}