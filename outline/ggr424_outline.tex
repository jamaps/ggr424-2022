% Don't touch this %%%%%%%%%%%%%%%%%%%%%%%%%%%%%%%%%%%%%%%%%%%
\documentclass[11pt]{article}
\usepackage{fullpage}
\usepackage[left=1in,top=1in,right=1in,bottom=1in,headheight=3ex,headsep=3ex]{geometry}
\usepackage{graphicx}
\usepackage{float}
\renewcommand{\theenumi}{\Alph{enumi})}

\newcommand{\blankline}{\quad\pagebreak[2]}
%%%%%%%%%%%%%%%%%%%%%%%%%%%%%%%%%%%%%%%%%%%%%%%%%%%%%%%%%%%%%%

% Modify Course title, instructor name, semester here %%%%%%%%

\title{\textbf{GGR424: Transportation Geography \& Planning}}
\author{Jeff Allen}
\date{Winter, 2022}

%%%%%%%%%%%%%%%%%%%%%%%%%%%%%%%%%%%%%%%%%%%%%%%%%%%%%%%%%%%%%%

% Don't touch this %%%%%%%%%%%%%%%%%%%%%%%%%%%%%%%%%%%%%%%%%%%
\usepackage[sc]{mathpazo}
\linespread{1.05} % Palatino needs more leading (space between lines)
\usepackage[T1]{fontenc}
\usepackage[mmddyyyy]{datetime}% http://ctan.org/pkg/datetime
\usepackage{advdate}% http://ctan.org/pkg/advdate
\newdateformat{syldate}{\twodigit{\THEMONTH}/\twodigit{\THEDAY}}
\newsavebox{\MONDAY}\savebox{\MONDAY}{Mon}% Mon
\newcommand{\week}[1]{%
	%  \cleardate{mydate}% Clear date
	% \newdate{mydate}{\the\day}{\the\month}{\the\year}% Store date
	\paragraph*{\kern-2ex\quad #1, \syldate{\today} - \AdvanceDate[4]\syldate{\today}:}% Set heading  \quad #1
	%  \setbox1=\hbox{\shortdayofweekname{\getdateday{mydate}}{\getdatemonth{mydate}}{\getdateyear{mydate}}}%
	\ifdim\wd1=\wd\MONDAY
	\AdvanceDate[7]
	\else
	\AdvanceDate[7]
	\fi%
}
\usepackage{setspace}
\usepackage{multicol}
%\usepackage{indentfirst}
\usepackage{fancyhdr,lastpage}
\usepackage{url}
\pagestyle{fancy}
\usepackage{hyperref}
\usepackage{lastpage}
\usepackage{amsmath}
\usepackage{layout}

\lhead{}
\chead{}
%%%%%%%%%%%%%%%%%%%%%%%%%%%%%%%%%%%%%%%%%%%%%%%%%%%%%%%%%%%%%%

% Modify header here %%%%%%%%%%%%%%%%%%%%%%%%%%%%%%%%%%%%%%%%%
\rhead{\footnotesize Course Outline | GGR424}
\lhead{\footnotesize Jeff Allen}
%%%%%%%%%%%%%%%%%%%%%%%%%%%%%%%%%%%%%%%%%%%%%%%%%%%%%%%%%%%%%%
% Don't touch this %%%%%%%%%%%%%%%%%%%%%%%%%%%%%%%%%%%%%%%%%%%
\lfoot{}
\cfoot{\small \thepage/\pageref*{LastPage}}
\rfoot{}

\usepackage{array, xcolor}
\usepackage{color,hyperref}
\definecolor{clemsonorange}{HTML}{f00000}
\hypersetup{colorlinks,breaklinks,linkcolor=clemsonorange,urlcolor=clemsonorange,anchorcolor=clemsonorange,citecolor=black}

\setlength{\parindent}{0em}
\setlength{\parskip}{0.8em}

\usepackage{colortbl}
\usepackage{tabularx,ragged2e}
\usepackage{sectsty}


\usepackage{helvet}

\begin{document}
	\allsectionsfont{\sffamily}
	
	
	\begin{center}
		
		|
		
		\Large{\textsf{\textbf{GGR424: Transportation Geography \& Planning}}}
		
		\normalsize
				
		Department of Geography \& Planning, University of Toronto
		
		Mondays 3pm to 5pm
		
		|
		
		
		
		\vspace{5 mm}
		
		\hrule
		
	\end{center}
		

	
	
	
	\vspace{3 mm}
	
	% First Section %%%%%%%%%%%%%%%%%%%%%%%%%%%%%%%%%%%%%%%%%%%%
	
	\section*{\textsf{Instructor}}
	
	Jeff Allen | \url{jeff.allen@utoronto.ca}
	
	Office Hours | Mondays 3:30pm-5:00pm or by appointment
	
	
	
	
	
	
	
	\section*{Course Description}
	
	% copied from posting
	Introductory overview of major issues in interurban and intraurban transportation at the local, national and international scale. 
	
	Topics include urban transportation, land use patterns and the environment, causes of and cures for congestion, public transit, infrastructure finance, and transport planning and policy setting. 
	
	
	\section*{Learning Outcomes}
	
	\begin{itemize}
		
		\item Understand fundamental concepts and theories in urban transportation geography and planning.
		
		\item Identify and critically assess major social, political, economic, and environmental issues related to urban transportation.
		
		\item Analyze and visualize transportation-related data (including using GIS) to describe travel behaviour, transportation networks, land use, and accessibility.
		
		\item Apply the theoretical and practical knowledge you have acquired from the course to create recommendations on improving urban transport systems.
		
	\end{itemize}
	
	
	
	
	
	
	

	
	
	
	% Second Section %%%%%%%%%%%%%%%%%%%%%%%%%%%%%%%%%%%%%%%%%%%
	
%	\section*{Prerequisites}
%	
%	There are no prerequisites for this courses. A background in GIS would be useful.
	
	
	
	
	
	
	
%	\section*{Course Objectives}
		
%	\begin{itemize}
%		\item Discuss the implications of the availability and suitability of crowdsourced geographic data for solving particular types of problems
%		\item Describe and critically evaluate potential problems and biases with data created by crowdsourced geographic projects.
%		\item Evaluate and describe potential implications (e.g., privacy, surveillance) caused by some of the technologies and socio-economic practices that support crowdsourcing.
%		\item Be able to contribute to OpenStreetMap as well as download OpenStreetMap data.
%		\item Learn how to create a simple crowdsourcing web application.
%		
%	\end{itemize}


	\section*{\textsf{Meeting Details}}
	
	Class will meet synchronously from 1:00pm-3:00pm on Mondays. For (at least) the first three weeks, class will meet online via zoom (links will be posted on the course website). Depending on U of T's COVID-19 policy, class may then shift to in-person in SS 2111

	
	
	\section*{Required Materials}
	
	There is no textbook for this course. All readings will be posted on Quercus. See the course schedule (below) for a list of readings.
	
	
	\section*{Course Communication}
	
	Please use email with discretion and mostly for personal matters that should not be shared with other members of the course (e.g., seeking an extension). 
	
	Please put “GGR424" in your subject line when emailing me so I can flag it. I will do my best to reply within 24 hours (not including weekends).
	
	For questions about course content, please save them for class when the answer can benefit other students. 
	
		
	\section*{Lateness \& Submissions Policy}
	
	All assignments must be submitted via Quercus. Except in the case of personal or medical emergencies, work must be submitted on time. Extensions may be permitted on a case-by-case basis through consultation with the instructor prior to the deadline. Late assignments will be docked 10\% per day, including weekends. Re-weighting of assignments/grades is not permitted.
	
	
	
	
		
	

	\section*{Projects \& Evaluation}
	
	
	\subsubsection*{Travel Field Notes (10\%)}
	
	Describe your personal impressions and experiences of a daily trip. This should be a regular/common trip that you currently do, or previously undertook regularly (e.g. your commute to school, to work, a trip to the grocery store, to the library, to visit family, etc.). First describe your trip including the mode(s) you used, the neighbourhoods you started and ended in, the main routes you travelled on (e.g. Highway 401, Bloor-Danforth Subway Line, etc.), and how long it took.
	% Then write about your feelings, perceptions, and observations taking various forms of transit, walking, cycling, and/or driving.
	Then evaluate aspects of this experience of urban transportation that 1) worked for you or were even enjoyable and 2) that were problematic, challenging, or stressful. For instance, in addition to travel times and congestion, you may want to think of issues like comfort, accessibility, equity, health, sustainability, and safety.	
	
	600 to 900 words + at least 2 photos relating to key locations in your description.
	
	\textit{Due January 24}
	
	
	
	
	\subsubsection*{Transit Station Critique (10\%)}
	
	Describe and critique a public transit station that you use (or have used) frequently. First, describe the transit routes that connect to it and the common modes of transport that people use to access the station. Second, critique the station in it's effectiveness in providing connections between routes and between modes, and in terms of it's public space and amenities. What parts of the station work well? Which do not? (e.g. this can include discussion on signage and wayfinding, benches and seating, cleanliness, crowding, accessibility considerations, other amenities, etc.). And third, briefly describe the land use around the station and argue whether its surrounding area is a good example of Transit Oriented Development (TOD). 
	
	600 to 900 words + at least 1 map of the station and surrounding area
	
	\textit{Due February 7}
	
	
	
	
	\subsubsection*{Transportation Data Analysis (30\%)}
	
	This assignment will consist of an analysis and interpretation of data from the Canadian census, the Transportation Tomorrow Survey (TTS), and various transportation datasets in GIS. Your assignment will be to query and map these data and answer a set of analytical questions. The final report will be approximately 5 to 7 pages and will include charts, graphs, and maps along with textual interpretation. Details about this assignment will be posted on the course website.
	
	\textit{Due February 28}
	
	
	
	\subsubsection*{Transportation Improvement Plan (Proposal 5\%, Report 25\%, Presentation 5\%)}
	
	You will propose an intervention, solution, or
	improvement to a transportation problem you have identified in either your previous assignments or another project of your choosing. This plan or proposal can vary by mode, geography, and objectives. You will be evaluated on how evidence-based and well-reasoned it is as well as how effectively it is communicated. Big and small ideas welcomed, as well as imaginative ones that address particular problems that exist in the transportation landscape (e.g. improvement for a particular street or transit line) or broader issues such as transportation equity, sustainability, potential mode shift, and efficiency. The plan should be approximately 5 to 7 pages. Feel free to consider using a variety of communication methods for your plan, including design sketches, maps, street plans, renderings, and/or photographs.
	
	The final two classes will be devoted to short (5 min) presentations followed by a brief Q\&A period. The report will be due after the last class, so the presentation does not need to consist of a "complete" project.
	
	Further details will be posted on the course website.
	
	
	\subsubsection*{In-Class Participation, Exercises, \& Quizzes (15\%)}
	
	The quality of your learning experience depends as much on your participation as well as what the instructor brings. There will be several interactive aspects to the class, including group discussions and design exercises. Attendance will be noted during class. Each of the first 11 weeks will involve at least one in-class activity, group discussion, or quiz. Each class will be worth 1.5\% (you thus have the opportunity to earn a maximum of 16.5\%).
	
	
	
	
	
	\section*{Academic Integrity}
	
	Academic integrity is essential to the pursuit of learning and scholarship in a university, and to ensuring that a degree from the University of Toronto is a strong signal of each student’s individual academic achievement. As a result, the University treats cases of cheating and plagiarism very seriously. The University of Toronto's Code of Behaviour on Academic Matters outlines the behaviours that constitute academic dishonesty and the processes for addressing academic offences. 
	
	For papers and assignments, this includes using someone else's ideas or words without appropriate acknowledgement, submitting your own work in more than one course without the permission of the instructor in all relevant courses,	making up sources or facts, or obtaining or providing unauthorized assistance on any assignment.
	
	All suspected cases of academic dishonesty will be investigated following procedures outlined in the Code of Behaviour on Academic Matters. If you have questions or concerns about what constitutes appropriate academic behaviour or appropriate research and citation methods, please reach out to me. For further information on academic integrity, please see \url{http://academicintegrity.utoronto.ca/}.
	
	
	\section*{Mental Health \& Accessibility}
	
	The University provides academic accommodations for students with disabilities in accordance with the terms of the Ontario Human Rights Code. This occurs through a collaborative process that acknowledges a collective obligation to develop an accessible learning environment that both meets the needs of students and preserves the essential academic requirements of the University's courses and programs. Please see \href{https://studentlife.utoronto.ca/department/accessibility-services/}{Accessibility Services} for more information.
	
	As a student at U of T, you may experience circumstances and challenges that can affect your academic performance and/or reduce your ability to participate fully in daily activities. An important part of the University experience is learning how and when to ask for help. There is no wrong time to reach out, which is why there are resources available for every situation and every level of stress. Here are some available resources:
	
	\begin{itemize}
		
		
		\item \href{https://safety.utoronto.ca/}{Student Life Safety \& Support}
		\item \href{https://studentlife.utoronto.ca/department/health-wellness/}{Student Life Health \& Wellness}
		\item \href{https://studentlife.utoronto.ca/task/support-when-you-feel-distressed/}{Emergency support if you’re feeling distressed}
		
		\item \href{https://mentalhealth.utoronto.ca/}{Student Mental Health Resources}
		
		\item  \href{https://geography.utoronto.ca/department/mental-health-resources/}{Additional mental health resources can also be found on the Geography website}
	\end{itemize}

	
	\section*{Writing}
	
	Clear writing and communicating is essential. You will be expected to write clearly and effectively on
	assignments. The University provides some resources through the writing centres. Brief advice on
	specific elements of writing for university courses can also be found here:	
	\url{https://writing.utoronto.ca/writing-centres/arts-and-science/}
	



	
	\section*{Course Schedule}
	
	The following outlines course topics and material on a week-to-week basis. For each week, there will be 1 or 2  readings (either accessible from a URL posted on the course website). For most weeks, I've also included links to other media (e.g. short videos, maps, interactive websites) that I encourage you to check out, as they will greatly enhance your learning experience throughout the course.

	
	\subsection*{Week 1 | Introduction to Transportation Geography \& Planning}
	
	\textit{January 10}

	Review of syllabus. Overview of key concepts and methods in transportation planning and geography. 
	
	% Class activity - mode prioritization
	% class quiz - class census
	
	

	
		
	
	\subsection*{Week 2 | Cars, Roads, \& Highways}
	
	\textit{January 17}
	
	Rise of cars as the predominant mode of travel in North American cities. How the infrastructure of automobiles has transformed transportation and land-use and has affected daily life. 
	
	\begin{itemize}
%		\item \textbf{Reading:} Urry, J. (2004). The ‘System’ of Automobility. Theory, Culture \& Society , 21 (4–5), 25–39. 	\url{https://doi.org/10.1177/0263276404046059}
		
		\item \textbf{Reading:} Jacobs, J. (1961). The Death and Life of Great American Cities. Vintage.
		Chapter 18: Erosion of cities or attrition of automobiles (will be posted online)
					
		% https://booksvooks.com/nonscrolablepdf/the-death-and-life-of-great-american-cities-pdf-1.html?page=4			
		
		\item \textbf{Reading:} Zipper, D. (2021). The Unstoppable Appeal of Highway Expansion. Bloomberg CityLab. \href{https://www.bloomberg.com/news/features/2021-09-28/why-widening-highways-doesn-t-bring-traffic-relief}{URL}
		
		\item (\textit{optional}) \textbf{Reading:} Robinson, R. (2011) The Spadina Expressway Controversy in Toronto, Ontario. \textit{ Canadian Historical Review} \href{https://www.utpjournals.press/doi/pdf/10.3138/chr.92.2.295}{URL}
		
		- -
		
		\item \textbf{Video:} Not Just Bikes (2021) Stroads. \href{https://www.youtube.com/watch?v=ORzNZUeUHAM}{URL}
		
		\item \textbf{Video:} TVO (2018) 
		Highways and our Transportation Future \href{https://www.tvo.org/video/highways-and-our-transportation-future}{URL}
		
		
	\end{itemize}

% https://www.newyorker.com/magazine/2019/07/29/was-the-automotive-era-a-terrible-mistake

	% Class Activity: Gardiner expressway options - tear down, maintain, replace with tunnel
	% or a 413, bypass debate?


	% Topics
	% History of cars, highways, and urban development (in Toronto)
	% Sprawl and Cars
	% Current prevlance of auto use (mode share)
	% Induced Demand
	% Accidents and Deaths
	% health impacts
	% waste o space - parking - 95% of time car is parked, parking lots massive amounts of hard space
	
	
	% Politics of the Car?


	\subsection*{Week 3 | Cycling \& Walking}
	
	\textit{January 24}
	
	Walking and cycling in the city. Health benefits of active travel. Safety issues. Streets as public space. Designing complete streets. 
	
	\begin{itemize}
		\item \textbf{Reading:} Hess, P. (2009) Avenues or Arterials: The struggle to change street building practices in Toronto, Canada. Journal of Urban Design, 14(1), 1 to 28. (will be posted online)
		
		- -
		
		\item \textbf{Interactive:} Explore Streetmix, an interactive tool for desiging cross-sections of streets. \href{https://streetmix.net}{URL}
		
		\item \textbf{Interactive:} Explore the City of Toronto's Vision Zero Plan \& Dashboard. \href{https://www.toronto.ca/services-payments/streets-parking-transportation/road-safety/vision-zero/vision-zero-dashboard/}{URL}
		
		\item \textbf{Interactive:} Explore The Centre for Active Transportation's (TCAT) website. \href{https://www.tcat.ca/}{URL}
		
		\item \textbf{Video:} Cycle Toronto (2021) Scarborough Needs a Cycling Network. \href{https://www.youtube.com/watch?v=gnwhc9fBa5k}{URL}
		
	\end{itemize}
	
	% Guest Lecture (TCAT?)
	
	% Class Activity - Ideas for redesigning a street in Toronto - Pick a street, redesign - then present to class?
	

	

		
	\subsection*{Week 4 | Public Transit}
	
	\textit{January 31}
	
	Theory and practice of public transportation planning, design, and operations. Overview of Transit Oriented Development (TOD)
	
	\begin{itemize}
		\item \textbf{Reading:} Walker, J. (2011) Human Transit. Chapters 1 and 2. pp.1 to 37 (will be posted online)
		
		\item (\textit{optional}) \textbf{Reading:} Ocejo, R. E., \& Tonnelat, S. (2014). Subway diaries: How people experience and practice riding the train. Ethnography, 15(4), 493–515.
		\href{https://doi.org/10.1177/1466138113491171}{URL}
		
		- -
		
		\item \textbf{Planning Doc.:} Review the TTC’s 5-Year Service Plan \& 10-Year Outlook \href{https://ttc-cdn.azureedge.net/-/media/Project/TTC/DevProto/Documents/Home/About-the-TTC/5_year_plan_10_year_outlook/Attachment-1-TTC_5_year_SP_web_accessible_R3.pdf?rev=69cfa3fbb3034d8a8ca5aaff03bf6a17&hash=9208204C7255C70154C0DFD161BA16F9}{PDF}, \href{https://www.ttc.ca/about-the-ttc/projects-and-plans/5-Year-Service-Plan-and-10-Year-Outlook}{URL}
		
		\item \textbf{Video:} Transit Analytics Lab (2020) History of Public Transit in Toronto \href{https://www.youtube.com/watch?v=0Rd4RzsRsWg}{URL}
		
		\item \textbf{Video:} RMTransit (2021) The Key to Building Transit-Oriented Development \href{https://www.youtube.com/watch?v=5HSI_PZBsPc}{URL}
		
		\item \textbf{Video:} Vox (2020) Why American public transit is so bad \href{https://www.youtube.com/watch?v=-ZDZtBRTyeI}{URL}
		
	\end{itemize}
	
	% Guest Lecture (MX? or TTC? - or none?) ??
	
	% Activity - overview / critique componetns of TTCs 5-year plan
	
	
	
	
	
	
	
	
	\subsection*{Week 5 | Transportation Networks \& Land Use}
	
	\textit{February 7}
	
	The spatial organization of transportation networks. Transportation and land-use. Measuring and evaluating accessibility.
	
	\begin{itemize}
		
		\item \textbf{Video Lecture:} Handy, S. (2019) Accessibility versus Mobility.  \href{https://www.youtube.com/watch?v=4OMv6CqrzvE}{URL}
		
		- - 
		
		\item \textbf{Map:} Explore The Neptis Geoweb Map \href{https://neptisgeoweb.org/}{URL}
		
		\item \textbf{Map:} Explore Geoapify's Commute Time Map \href{https://commutetimemap.com/}{URL}
		
		\item \textbf{Map:} Explore Map Showing Access to Employment in
		Canadian Cities. \href{https://sausy-lab.github.io/canada-transit-access/map.html}{URL}
		
		% something on modal disparaties
		
		% something on 15 minute cities
		
	\end{itemize}
	
	% Activity / Quiz, measure Accessibility - conceptual ideas and answers
	
	
	
	
	
	\subsection*{Week 6 | Maps, Data, \& GIS}
	
	\textit{February 14}
		
	Common forms of data used for analyzing transportation systems. GIS analysis for transportation planning and research. Transportation data visualization and cartography.
	
	
	\begin{itemize}
		
		\item \textbf{Reading:} Guo, Z. (2011). Mind the map! The impact of transit maps on path choice in public transit. Transportation Research Part A: Policy and Practice, 45(7), 625 to 639. (will be posted online)
		
		- -
		
		\item \textbf{Maps:} Explore maps on the Transit Maps Blog \href{http://transitmap.net/}{URL}
		
		\item \textbf{Interactive Story:} Barry, M. \& Card, B. (2014) Visualizing MBTA Data. \href{http://mbtaviz.github.io/}{URL}
		
		\item \textbf{Interactive Story:} The Pudding (2017) How far is too far? An analysis of driving times to abortion clinics in the US. \href{https://pudding.cool/2017/09/clinics/}{URL}
		
		\item \textbf{GIS:} Download and familiarize yourself with QGIS (if you do not have it already). \href{https://www.qgis.org/en/site/index.html}{URL} 
		
		
	\end{itemize}
	
	% Guest Lecture - Nate, bike map navigation stuff
	
	% find a map, or transportation vizualization, post it on the discussion - a couple sentances what you like and do not like
	
	
	
	\subsection*{Week 7 | Reading Week}
	
	\textit{February 21}
	
	
	
	\subsection*{Week 8 | Sustainability}
	
	\textit{February 28}
	
	Urban transportation and climate change. Potential of electric vehicles. Incentives for shifting to more sustainable modes.
	
	% A2 - Transportation Data Analysis Due
	
	\begin{itemize}
		\item \textbf{Reading:} Banister, David. (2011). Cities, mobility and climate change. Journal of Transport
		Geography, 19(6), 1538 to 1546. \href{https://doi.org/10.1016/j.jtrangeo.2011.03.009}{URL}
		
		- -
		
		\item \textbf{Interactive:} Our World in Data (2022) Transport \href{https://ourworldindata.org/transport}{URL}
	\end{itemize}

	% Guest Lecture (Pembina?)
	
	% activity? quiz?
	
	
	
	\subsection*{Week 9 | Health \& Equity}
	
	\textit{March 7}
	
	How the costs and benefits of transportation are (in)equitably distributed. Health impacts of transportation (e.g. pollution, noise).
	
	\begin{itemize}
		\item \textbf{Reading:} Lucas, K. (2012). Transport and social exclusion: Where are we now?. Transport policy, 20, 105 to 113. \href{https://doi.org/10.1016/j.tranpol.2012.01.013}{URL}
		
		\item \textit{(Optional)} \textbf{Reading:} Litman, T. (2020), Evaluating Transportation Equity Guidance For Incorporating Distributional Impacts in Transportation Planning, Victoria Transport Policy Institute. \href{https://vtpi.org/equity.pdf}{URL}
		
		- - 
		
		\item \textbf{Map:} Farber, S. \& Allen, J. (2018) Transit in Toronto: Connections between socioeconomic status and transit availability.  \href{http://edu.maps.arcgis.com/apps/Cascade/index.html?appid=58618c037f344aaaada20b0c894e011c}{URL}
		
		\item \textbf{Map:} TransitCenter (2021) TransitCenter Equity Dashboard. \href{https://dashboard.transitcenter.org/}{URL}
		
		
	\end{itemize}
	
	% Guest Lecture (MJ - Matt, Alex, C/P, 2/3 people sessions)
	
	
	
	
	\subsection*{Week 10 | Economics \& Politics}
	
	\textit{March 14}
	
	Financing transportation infrastructure. Transportation evidence and decision-making. Cost-benefit analyses. 
	
	\begin{itemize}
		
		\item \textbf{Reading:} Pagliaro, J. \& Spurr, B. (2017) How politics, not evidence, drives transit planning in Toronto. \href{https://www.thestar.com/news/city_hall/2017/09/18/how-politics-not-evidence-drives-transit-planning-in-toronto.html}{URL}
		
		\item \textbf{Reading:} Metrolinx (2020) Ontario Line Preliminary Design 
		Business Case Summary.
		\href{https://www.metrolinx.com/en/regionalplanning/projectevaluation/benefitscases/benefits_case_analyses.aspx}{URL}
		
		
	\end{itemize}
	
	% Guest Lecture - (MX? else?)
	
	
	
		
	\subsection*{Week 11 | The Future of Transportation}
	
	\textit{March 21}
	
	Potential and concerns with new mobility technologies, such as shared mobility (e.g. ride hailing, bike-share, etc.), on-demand transit, and autonomous vehicles.
	
	\begin{itemize}
		\item \textbf{Reading:} Millard-Ball, A. (2018). Pedestrians, Autonomous Vehicles, and Cities. Journal of
		Planning Education and Research, 38(1), 6 to 12. (will be posted online)
		
		\item \textbf{Reading:} Zhang, Y., Farber, S., \& Young, M. (2021). Eliminating barriers to nighttime activity participation: the case of on-demand transit in Belleville, Canada. Transportation, 1 to 24. (will be posted online)
	\end{itemize}
	
	% guest lecture - M? other researcher / grad student?
	
	% activity - plan a road for AVs/// maybe
	
	
	
	
	\subsection*{Week 12 | Presentations}
	
	\textit{March 28}
	
	
	\subsection*{Week 13 | Presentations}

	\textit{April 4}

	
	


	
\end{document}